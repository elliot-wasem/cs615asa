%traceroute -a
%
%traceroute -w 1 www.uni-marburg.de | awk '{print $3}' | grep "(" | sed -e 's/(//' -e 's/)//' | xargs -n 1 whois -h whois.cymru.com
%

\documentclass[xga]{xdvislides}
\usepackage[landscape]{geometry}
\usepackage{graphics}
\usepackage{graphicx}
\usepackage{colordvi}

\begin{document}
\setfontphv

%%% Headers and footers
\lhead{\slidetitle}                               % default:\lhead{\slidetitle}
\chead{CS615 - Aspects of System Administration}% default:\chead{\relax}
\rhead{Slide \thepage}                       % default:\rhead{\sectiontitle}
\lfoot{\Gray{DNS; HTTP}}% default:\lfoot{\slideauthor}
\cfoot{\relax}                               % default:\cfoot{\relax}
\rfoot{\Gray{\today}}

\newcommand{\smallish}{\fontsize{15}{20}\selectfont}

\vspace*{\fill}
\begin{center}
	\Hugesize
		CS615 - Aspects of System Administration\\ [1em]
		DNS; HTTP\\ [1em]
	\hspace*{5mm}\blueline\\ [1em]
	\Normalsize
		Department of Computer Science\\
		Stevens Institute of Technology\\
		Jan Schaumann\\
		\verb+jschauma@stevens-tech.edu+
		\verb+https://stevens.netmeister.org/615/+
\end{center}
\vspace*{\fill}

\subsection{Current Events}

\$10,000 Scholarship opportunity: \\

"The North American Network Operators' Group (NANOG),
is the professional association for Internet
engineering, architecture and operations. Our core
focus is on continuous improvement of the data
transmission technologies, practices, and facilities
that make the Internet function. In an effort to
support the next generation of network operators,
NANOG has established a scholarship program to assist
current undergraduate and graduate-level students
pursuing a degree in one of the eligible fields listed
below." \\

\verb+https://www.scholarsapply.org/nanog/+

\subsection{Current Events}
\vspace*{\fill}
\begin{center}
	\includegraphics[scale=0.7]{pics/tzmap-namerica.eps} \\
	\verb+https://www.timeanddate.com/time/dst/events.html+
\end{center}
\vspace*{\fill}

\subsection{Current Events}
\Huge
\begin{center}
Falsehoods Programmers Believe About Time: \\

\vspace{.5in}
1. There are always 24 hours in a day. \\

\vspace{.5in}
\verb+http://FalsehoodsAboutTime.com+
\end{center}
\Normalsize

\subsection{In the beginning...}
\vspace*{\fill}
\begin{center}
	\includegraphics[scale=0.8]{pics/2computers.eps} \\
\end{center}
\vspace*{\fill}

\subsection{In the beginning...}
\vspace*{\fill}
\begin{center}
	\includegraphics[scale=0.8]{pics/2computers-nic.eps} \\
\end{center}
\vspace*{\fill}

\subsection{In the beginning...}
\vspace*{\fill}
\begin{center}
	\includegraphics[scale=0.8]{pics/3computers.eps} \\
\end{center}
\vspace*{\fill}

\subsection{In the beginning...}
\vspace*{\fill}
\begin{center}
	\includegraphics[scale=0.8]{pics/3computers-1.eps} \\
\end{center}
\vspace*{\fill}

\subsection{In the beginning...}
\vspace*{\fill}
\begin{center}
	\includegraphics[scale=0.8]{pics/3computers-2.eps} \\
\end{center}
\vspace*{\fill}

\subsection{In the beginning...}
\vspace*{\fill}
\begin{center}
	\includegraphics[scale=0.8]{pics/arpanet1.eps} \\
	{\tt https://is.gd/DdPNCo} \\
\end{center}
\vspace*{\fill}


\subsection{In the beginning...}
\begin{verbatim}
# Host Database
# This file should contain the addresses and aliases
# for local hosts that share this file.
#
127.0.0.1               localhost localhost.
#
# RFC 1918 specifies that these networks are "internal".
# 10.0.0.0      10.255.255.255
# 172.16.0.0    172.31.255.255
# 192.168.0.0   192.168.255.255
10.0.0.1	UCLA-TEST
10.0.0.2	SRI-SPRM
10.0.0.4	UTAH-CS
\end{verbatim}


\subsection{But then...}
\vspace*{\fill}
\begin{center}
	\includegraphics[scale=0.3]{pics/routed.eps} \\
\end{center}
\vspace*{\fill}

\subsection{The Domain Name System}
\vspace{.5in}
\begin{center}
	\Huge
	Computers like numbers. \\
\vspace{.5in}
\begin{verbatim}
         10011011111101100101100110011111
\end{verbatim}
\end{center}
\Normalsize

\subsection{The Domain Name System}
\vspace{.5in}
\begin{center}
	\Huge
	Computers like numbers. \\
\vspace{.5in}
\begin{verbatim}
      10011011  11110110  01011001  10011111

        155   .   246   .    89   .   159
\end{verbatim}
\end{center}
\Normalsize

\subsection{The Domain Name System}
\vspace{.5in}
\begin{center}
	\Huge
	People like names. \\
\vspace{.5in}
\verb+ash.cs.stevens-tech.edu+
\end{center}
\Normalsize


\subsection{The Domain Name System}
\vspace*{\fill}
\begin{center}
	\includegraphics[scale=0.6]{pics/phonebook.eps}
\end{center}
\vspace*{\fill}

\subsection{The New Phonebook is here!}
\vspace*{\fill}
\begin{center}
	\verb+https://is.gd/XXp2sC+ \\
	\addvspace{.5in}
	\verb+wget -q -O - https://is.gd/XXp2sC | grep -c "^HOST"+
\end{center}
\vspace*{\fill}

\subsection{DNS: A distributed database}
\vspace*{\fill}
\begin{center}
	\includegraphics[scale=0.75]{pics/distributed-database.eps}
\end{center}
\vspace*{\fill}

\subsection{The Domain Name Space}
\vspace{.5in}
\begin{center}
	\Huge
	The domain name space consists of a tree of {\em domain} names.
\end{center}
\Normalsize

\subsection{DNS: A hierarchical system}
\vspace*{\fill}
\begin{center}
	\includegraphics[scale=0.75]{pics/hierarchical-dns.eps}
\end{center}
\vspace*{\fill}

\subsection{The Domain Name Space}
\vspace{.5in}
\begin{center}
	\Huge
	The domain name space consists of a tree of {\em domain} names. \\
	\vspace{.5in}
	A subtree divides into {\em zones}.
\end{center}
\Normalsize

\subsection{The Domain Name Space}
\vspace{.5in}
\begin{center}
	\Huge
	The domain name space consists of a tree of {\em domain} names. \\
	\vspace{.5in}
	A subtree divides into {\em zones}. \\
	\vspace{.5in}
	Each node may contain {\em resource records}.
\end{center}
\Normalsize

\subsection{The Domain Name Space}
\vspace*{\fill}
\begin{center}
	\includegraphics[scale=0.74]{pics/dns-space.eps}
\end{center}
\vspace*{\fill}

\subsection{Domain Names}
\vspace{.5in}
\begin{center}
	\Huge
	\verb+ash.cs.stevens-tech.edu+ \\
	\vspace{.5in}
	Domain Names are read from right to left and components separated by a ``\verb+.+''.
\end{center}
\Normalsize

\subsection{Domain Names}
\vspace{.5in}
\begin{center}
	\Huge
	\verb+ash.cs.stevens-tech.edu.+ \\
	\vspace{.5in}
	The {\em root} is known as ``\verb+.+'', but is usually left out.
\end{center}
\Normalsize

\subsection{Domain Names}
\vspace{.5in}
\begin{center}
	\Huge
	\verb+ash.cs.stevens-tech.+{\bf edu}\verb+.+ \\
	\vspace{.5in}
	There is a small number of {\em top level domains}.
\end{center}
\Normalsize

\subsection{Domain Names}
\vspace{.5in}
\begin{center}
	\Huge
	\verb+ash.cs.stevens-tech.+{\bf edu}\verb+.+ \\
	\vspace{.5in}
	There is a number of {\em top level domains}. \\
	\vspace{.5in}
	\Normalsize
	\begin{verbatim}
wget -O - ftp://rs.internic.net/domain/root.zone | \
        grep "IN<tab>*NS<tab>" | awk '{print $1}' | sort -u | wc -l
\end{verbatim}
	\vspace{.25in}
	\verb+https://data.iana.org/TLD/tlds-alpha-by-domain.txt+ \\
	\verb+https://en.wikipedia.org/wiki/List_of_Internet_top-level_domains+
\end{center}
\Normalsize


\subsection{Domain Names}
\vspace{.5in}
\begin{center}
	\Huge
	\verb+ash.cs.+{\bf stevens-tech}\verb+.edu.+ \\
	\vspace{.5in}
	Each {\em domain} can be divided into any number of {\em sub domains}.
\end{center}
\Normalsize

\subsection{Domain Names}
\vspace{.5in}
\begin{center}
	\Huge
	\verb+ash.+{\bf cs}\verb+.stevens-tech.edu.+ \\
	\vspace{.5in}
	Each {\em domain} can be divided into any number of {\em sub domains}.
\end{center}
\Normalsize

\subsection{Domain Names}
\vspace{.5in}
\begin{center}
	\Huge
	{\bf ash}\verb+.cs.stevens-tech.edu.+ \\
	\vspace{.5in}
	The left-most component of a domain name may be a {\em hostname}.
\end{center}
\Normalsize

\subsection{Fully Qualified Domain Names}
\vspace{.5in}
\begin{center}
	\Huge
	\verb+ash.cs.stevens-tech.edu.+ \\
	\vspace{.5in}
	A {\em hostname} with a domain name is known as a {\em FQDN}.
\end{center}
\Normalsize

\subsection{The Original IANA}
\vspace*{\fill}
\begin{center}
	\includegraphics[scale=1.25]{pics/postel.eps}
\end{center}
\vspace*{\fill}

\subsection{NIC and Network Solutions}
Before the DNS, the Network Information Center (NIC)
at Stanford Research Institute (SRI) allocated domain
names. IANA (effectively: Jon Postel) assigned, NIC
published.  \\

{\tt https://www.internic.net} \\

In 1991, this was contracted out to Network Solutions,
Inc. (NSI), which held the monopoly on DNS
registrations (within .com, .org, .mil, .gov, .edu, and
.net) until around 1998. \\


\subsection{Registries}
IANA manages the root zone (.), arpa.; gTLD registries
handle gTLDs, ccTLD registries handle ccTLDs.  ICANN
accredits {\em domain name registries}. \\

Registries
\begin{itemize}
	\item may function as a Domain Name Registrar
	\item may delegate Domain Name registration
	\item control policies of allocations
	\item can (and do) censor, revoke, change, ... entries (e.g. {\tt vb.ly})
\end{itemize}

\vspace{.5in}
The domain name space is a tree; if you control one
node, you control all the branches and subtrees.

\subsection{DNS servers come in two flavors}
\vspace*{\fill}
\begin{center}
	\begin{tabular}{ c c c }
	\includegraphics[scale=1.5]{pics/vanilla.eps} & \hspace{.5in} & \includegraphics[scale=1.5]{pics/chocolate.eps} \\
	\hspace{.3in} \Huge Authoritative & & \hspace{.3in} \Huge Recursive \\
	\hspace{.3in} \Huge Nameservers & & \hspace{.3in} \Huge Nameservers \\
	\end{tabular}
\end{center}
\vspace*{\fill}

\subsection{Hostname resolution}
Resolution on a recursive nameserver (aka {\em resolver}) involves a number of queries:
\vspace{.5in}
\begin{verbatim}
$ nslookup ash.cs.stevens-tech.edu
Server:         127.0.0.1
Address:        127.0.0.1#53

Non-authoritative answer:
Name:   ash.cs.stevens-tech.edu
Address: 155.246.89.159

$
\end{verbatim}

\subsection{Hostname resolution}
Resolution on a {\em resolver} involves a number of queries:
\begin{verbatim}
IP panix.netmeister.org.62105 > i.root-servers.net.domain:
        11585 [1au] A? ash.cs.stevens-tech.edu. (52)
IP i.root-servers.net.domain > panix.netmeister.org.62105:
        11585- 0/8/8 (494)
IP panix.netmeister.org.53168 > a.gtld-servers.net.domain:
        46575 [1au] A? ash.cs.stevens-tech.edu. (52)
IP a.gtld-servers.net.domain > panix.netmeister.org.53168:
        46575- 0/6/3 (609)
IP panix.netmeister.org.41071 > nrac.stevens-tech.edu.domain:
        24322 [1au] A? ash.cs.stevens-tech.edu. (52)
IP nrac.stevens-tech.edu.domain > panix.netmeister.org.41071:
        24322*- 1/2/3 A[|domain]
\end{verbatim}
\Normalsize

\subsection{Hostname resolution}
Resolution on a {\em resolver} involves a number of queries:
\begin{verbatim}
$ host -t ns .
. name server I.ROOT-SERVERS.NET.
. name server D.ROOT-SERVERS.NET.
. name server C.ROOT-SERVERS.NET.
. name server M.ROOT-SERVERS.NET.
. name server F.ROOT-SERVERS.NET.
. name server A.ROOT-SERVERS.NET.
. name server E.ROOT-SERVERS.NET.
. name server L.ROOT-SERVERS.NET.
. name server H.ROOT-SERVERS.NET.
. name server J.ROOT-SERVERS.NET.
. name server B.ROOT-SERVERS.NET.
. name server G.ROOT-SERVERS.NET.
. name server K.ROOT-SERVERS.NET.
$
\end{verbatim}

\subsection{Hostname resolution}
Resolution on a {\em resolver} involves a number of queries:
\begin{verbatim}
$ dig -t ns edu.
[...]
;; ANSWER SECTION:
edu.                    172800  IN      NS      l.edu-servers.net.
edu.                    172800  IN      NS      f.edu-servers.net.
edu.                    172800  IN      NS      c.edu-servers.net.
edu.                    172800  IN      NS      g.edu-servers.net.
edu.                    172800  IN      NS      a.edu-servers.net.
edu.                    172800  IN      NS      d.edu-servers.net.

;; ADDITIONAL SECTION:
c.edu-servers.net.      36626   IN      A       192.26.92.30
d.edu-servers.net.      13274   IN      A       192.31.80.30
l.edu-servers.net.      36626   IN      A       192.41.162.30
[...]
$
\end{verbatim}
\Normalsize

\subsection{Hostname resolution}
Resolution on a {\em resolver} involves a number of queries:
\begin{verbatim}
$ dig @c.edu-servers.net -t ns stevens.edu.
[...]
;; AUTHORITY SECTION:
stevens.edu.            172800  IN      NS      nrac.stevens-tech.edu.
stevens.edu.            172800  IN      NS      sitult.stevens-tech.edu.

;; ADDITIONAL SECTION:
nrac.stevens-tech.edu.  172800  IN      A       155.246.1.21
sitult.stevens-tech.edu. 172800 IN      A       155.246.1.20
[...]
$
\end{verbatim}

\subsection{Hostname resolution}
\vspace*{\fill}
\begin{center}
	\includegraphics[scale=0.9]{pics/resolution.eps}
\end{center}
\vspace*{\fill}


\subsection{Hostname resolution}
Resolution on a {\em resolver} involves a number of queries:
\begin{verbatim}
$ nslookup ash.cs.stevens-tech.edu
Server:         127.0.0.1
Address:        127.0.0.1#53

Non-authoritative answer:
Name:   ash.cs.stevens-tech.edu
Address: 155.246.89.159

$
\end{verbatim}

\subsection{Hostname resolution}
\vspace*{\fill}
\begin{center}
	\includegraphics[scale=0.4]{pics/chicken-egg.eps} \\
	\vspace*{\fill}
\end{center}

\subsection{Hostname resolution}
\vspace*{\fill}
\begin{center}
	\includegraphics[scale=0.4]{pics/chicken-egg.eps} \\
	\addvspace{.2in}
	\verb+$ ftp -o - ftp.internic.net:/domain/db.cache | more+ \\
	\verb+https://www.internic.net/zones/named.root+
	\vspace*{\fill}
\end{center}

\subsection{Operation Global Blackout}
\vspace*{\fill}
\begin{center}
	\includegraphics[scale=0.8]{pics/anonymous.eps} \\
	\addvspace{.2in}
	\verb+https://pastebin.com/XZ3EGsbc+ \\
	\addvspace{.1in}
\end{center}
\vspace*{\fill}

\subsection{DNS: A distributed system}
\vspace{.5in}
\begin{center}
	\Huge
	There are 13 \verb+root+ servers. \\
\end{center}
\Normalsize

\subsection{DNS: A distributed system}
\vspace{.5in}
\begin{center}
	\Huge
	There are 13 \verb+root+ servers. \\
	\vspace{.5in}
	Except... there are more.
\end{center}
\Normalsize

\subsection{DNS: A distributed system}
\vspace{.5in}
\begin{center}
	\Huge
	There are 13 \verb+root+ {\em authorities}. \\
\end{center}
\Normalsize

\subsection{DNS: A distributed system}
\vspace{.5in}
\begin{center}
	\Huge
	There are 13 \verb+root server+ {\em addresses}. \\
\end{center}
\Normalsize

\subsection{DNS: A distributed system}
\vspace{.5in}
\begin{center}
	\Huge
	There are hundreds of \verb+root+ servers. \\
\end{center}
\Normalsize

\subsection{DNS: A distributed system}
\vspace*{\fill}
\begin{center}
	\includegraphics[scale=0.5]{pics/root-servers.eps}
\end{center}
\vspace*{\fill}
See e.g.: {\tt https://e.root-servers.org/}

\subsection{Operation Global Blackout}
\vspace*{\fill}
\begin{center}
	\includegraphics[scale=0.8]{pics/anonymous-tweet.eps} \\
\end{center}
\vspace*{\fill}


\subsection{DNS: A distributed database}
\vspace*{\fill}
\begin{center}
	\includegraphics[scale=0.75]{pics/distributed-database.eps}
\end{center}
\vspace*{\fill}


\subsection{DNS Resource Records}
More than just {\tt A} and {\tt AAAA}:
\begin{itemize}
	\item {\em CAA} -- certificate authority authorization
	\item {\em CNAME} -- the canonical name for an alias
	\item {\em MX} -- mail exchange
	\item {\em NS} -- an authoritative name server
	\item {\em SOA} -- marks the start of a zone of authority
	\item {\em SRV} -- service locator (e.g. for kerberos)
	\item {\em PTR} -- a domain name pointer
	\item {\em TXT} text strings
	\item ...
\end{itemize}

\subsection{DNS Resource Records}
You've all seen PTR records:
\\

\begin{verbatim}
$ host ash.cs.stevens-tech.edu
ash.cs.stevens-tech.edu has address 155.246.89.159
ash.cs.stevens-tech.edu mail is handled by 0 guinness.cs.stevens-tech.edu.
$ host 155.246.89.159
159.89.246.155.in-addr.arpa domain name pointer ash.cs.stevens-tech.edu.
$ 
\end{verbatim}

Stevens doesn't have write access to the {\tt
in-addr.arpa} domain.  How does this work?
 
\subsection{Creative uses of DNS Resource Records}
\begin{itemize}
	\item identifying sources of SPAM (via e.g. an RBL)
	\item detect email spoofing (via e.g. SPF)
	\item find out if the internet is on fire: \\
		\verb|dig +short txt istheinternetonfire.com|
	\item find ASN numbers by IP addresses: \\
		\verb|dig +short 159.89.246.155.origin.asn.cymru.com TXT|
	\item check a resolver's source port randomization (to help
		mitigate DNS Cache Poisoning attacks): \\
		\verb|dig +short porttest.dns-oarc.net TXT|
	\item using DNS to publish SSH key fingerprints (RFC4255,
ssh\_config(5) \verb+VerifyHostKeyDNS+; for best results combine with DNSSEC)
\end{itemize}

\subsection{DNS Implications}

\begin{itemize}
	\item information from the DNS is used for authentication,
		authorization, and as a source of truth
	\item DNSSEC is not widely deployed and
		carries implementation challenges
	\item DNS traffic is ubiquitous, may escape
		ACLs and restrictions
	\item faulty information can lead to
		unexpected and difficult to troubleshoot
		failures
	\item TTLs and caches can prolong outages as
		you wait for propagation of changes
	\item if you pwn the DNS, you pwn the entire
		target (hey, let's attack the registrar!)
	\item any time you outsource something, you
		lose control; any time you own solving
		a problem, you assert that you know
		how to solve this better than others
\end{itemize}

\newpage
\vspace*{\fill}
\begin{center}
    \Hugesize
        Hooray! \\ [1em]
    \hspace*{5mm}
    \blueline\\
    \hspace*{5mm}\\
        5 Minute Break
\end{center}
\vspace*{\fill}

\newpage
\vspace*{\fill}
\begin{center}
	\Hugesize
		Hypertext Transfer Protocol\\ [1em]
	\hspace*{5mm}
	\blueline\\
	\hspace*{5mm}\\
		Today's Universal Internet Pipe
\end{center}
\vspace*{\fill}

%\subsection{Set up an HTTP server.}
%
%\vspace*{\fill}
%Start an EC2 instance and set up an HTTP server to listen on port 8080.
%Add a simple index.html file containing your username.
%
%When done, paste the full URL (ie http:///) into the class IRC channel
%\#cs615asa.
%
%{\tt https://webchat.freenode.net/}
%\vspace*{\fill}

\subsection{HTTP: Hypertext}
\vspace{.5in}
\begin{center}
	\Huge
	W W W
	\\
\vspace{.5in}
	{\em ``The World Wide Web is the only thing I know of whose shortened form
	takes three times longer to say than what it's short for.'' -- Douglas Adams}
\end{center}
\Normalsize


\subsection{HTTP: Hypertext}
\begin{center}
	\includegraphics[scale=0.9]{pics/http-proposal-detail.eps} \\
	\vspace{.5in}
	\verb+https://is.gd/JnZaN6+
\end{center}

\subsection{HTTP}
\vspace{.5in}
\begin{center}
	\Huge
	Hypertext Transfer Protocol
	\\
	\vspace{.5in}
	RFC2616
\end{center}
\Normalsize

\subsection{HTTP}
\vspace{.5in}
\begin{center}
	\Huge
	HTTP is a request/response protocol.
\end{center}
\Normalsize

\subsection{The Hypertext Transfer Protocol}
HTTP is a request/response protocol:
\begin{enumerate}
	\item client sends a request to the server
	\item server responds
\end{enumerate}

\subsection{The Hypertext Transfer Protocol}
HTTP is a request/response protocol:
\begin{enumerate}
	\item client sends a request to the server
		\begin{itemize}
			\item request method
			\item URI
			\item protocol version
			\item request modifiers
			\item client information
		\end{itemize}
	\item server responds
\end{enumerate}

\subsection{HTTP: A client request}
\vspace*{.5in}
\\
\Hugesize
\begin{center}
\begin{verbatim}
$ telnet www.google.com 80
Trying 173.194.75.147...
Connected to www.google.com.
Escape character is '^]'.
GET / HTTP/1.0
\end{verbatim}
\end{center}
\Normalsize
\vspace*{\fill}


\subsection{The Hypertext Transfer Protocol}
HTTP is a request/response protocol:
\begin{enumerate}
	\item client sends a request to the server
		\begin{itemize}
			\item request method
			\item URI
			\item protocol version
			\item request modifiers
			\item client information
		\end{itemize}
	\item server responds
		\begin{itemize}
			\item status line (including success or error code)
			\item server information
			\item entity metainformation
			\item content
		\end{itemize}
\end{enumerate}

\subsection{HTTP: a server response}
%\smallish
\begin{verbatim}
HTTP/1.0 200 OK
Date: Sun, 31 Mar 2013 01:54:40 GMT
Set-Cookie: PREF=ID=c5eb56d629b347cc:FF=0:TM=1364694880:LM=1364694880:
S=sIdRFdxV9YvtQOlG; expires=Tue, 31-Mar-2015 01:54:40 GMT; path=/;
domain=.google.com
Set-Cookie: NID=67=hvBnOob2NoZW4haTJVfajbcyn_jips50lKRe-8nawzdCZ6AukNR
_s8CNHD6ZA-Z2721nA3TpLrNXt-2zyIui23j4kdsdF8Gg--PmGsMOJ3Jv5frEzQG1elHJv92HL-w2;
expires=Mon, 30-Sep-2013 01:54:40 GMT; path=/; domain=.google.com; HttpOnly
Server: gws

<!doctype html><html itemscope="itemscope" itemtype="http://schema.org/WebPage">
<head><meta content="Search the...

\end{verbatim}
%\Normalsize

\subsection{The Hypertext Transfer Protocol}
Server status codes:
\begin{itemize}
	\item {\bf 1xx} -- Informational; Request received, continuing process
	\item {\bf 2xx} -- Success; The action was successfully received,
        understood, and accepted
	\item {\bf 3xx} -- Redirection; Further action must be taken in order to
        complete the request
	\item {\bf 4xx} -- Client Error; The request contains bad syntax or
		cannot be fulfilled
	\item {\bf 5xx} -- Server Error; The server failed to fulfill an
		apparently valid request
\end{itemize}

\subsection{HTTP: A client request}
\smallish
\begin{verbatim}
$ telnet www.cs.stevens.edu 80
Trying 155.246.89.84...
Connected to www.cs.stevens-tech.edu.
Ecape character is '^]'.
GET / HTTP/1.0

HTTP/1.1 301 Moved Permanently
Date: Mon, 05 Mar 2018 20:41:06 GMT
Server: Apache
Location: https://www.cs.stevens.edu/
Vary: Accept-Encoding
Content-Length: 235
Connection: close
Content-Type: text/html; charset=iso-8859-1
\end{verbatim}
\Normalsize

\subsection{HTTP: A client request}
\smallish
\begin{verbatim}
$ printf "HEAD / HTTP/1.1\r\nHost: www.cs.stevens.edu\r\n\r\n" |
        openssl s_client -quiet -ign_eof -connect www.cs.stevens.edu:443 2>/dev/null

HTTP/1.1 302 Found
Date: Mon, 05 Mar 2018 20:53:38 GMT
Server: Apache
Location: https://www.stevens.edu/ses/cs
Vary: Accept-Encoding
Content-Type: text/html; charset=iso-8859-1
\end{verbatim}
\Normalsize

\subsection{HTTP: A client request}
\smallish
\begin{verbatim}
$ printf "HEAD /ses/cs HTTP/1.1\r\nHost: www.stevens.edu\r\n\r\n" |
        openssl s_client -quiet -ign_eof -connect www.stevens.edu:443 2>/dev/null

HTTP/1.1 301 Moved Permanently
Date: Mon, 05 Mar 2018 20:54:51 GMT
Content-Type: text/html; charset=UTF-8
Location: https://www.stevens.edu/schaefer-school-engineering-science/departments/computer-science

\end{verbatim}
\Normalsize


\subsection{HTTP: A client request}
\smallish
\begin{verbatim}
$ printf "HEAD /schaefer-school-engineering-science/departments/computer-science HTTP/1.1\r\nHost: www.stevens.edu\r\n\r\n" |
        openssl s_client -quiet -ign_eof -connect www.stevens.edu:443 2>/dev/null 

HTTP/1.1 200 OK
Date: Mon, 05 Mar 2018 20:56:37 GMT
Content-Type: text/html; charset=utf-8
Connection: keep-alive
Expires: Sun, 19 Nov 1978 05:00:00 GMT
Last-Modified: Mon, 05 Mar 2018 16:44:39 GMT
[...]
\end{verbatim}
\Normalsize

\subsection{HTTP: A client request}
\begin{center}
	\includegraphics[scale=0.4]{pics/www.cs.stevens.edu.eps}
\end{center}


\subsection{HTTP - more than just text}
HTTP is a {\em Transfer Protocol} -- serving {\em data}, not any specific
text format.

\begin{itemize}
	\item {\tt Accept-Encoding} client header can specify different formats
		such as {\tt gzip} or {\tt deflate} for compression etc.
		communications, etc.
	\item corresponding server headers: {\tt Content-Type} and
		{\tt Content-Encoding}
\end{itemize}
\begin{center}
	\includegraphics[scale=2.0]{pics/datatransfer.eps}
\end{center}

\subsection{HTTP - more than just static data}
HTTP is a {\em Transfer Protocol} -- what is transferred need not be
static; resources may generate different data to return based on many
variables.

\begin{itemize}
	\item CGI -- resource is {\em executed}, needs to generate
		appropriate response headers
	\item server-side scripting (ASP, PHP, Perl, ...)
	\item client-side scripting (JavaScript/ECMAScript/JScript,...)
	\item applications based on HTTP, using:
		\begin{itemize}
			\item AJAX
			\item RESTful services
			\item JSON, XML, YAML to represent state and
				abstract information
		\end{itemize}
\end{itemize}

\subsection{HTTP Proxy Servers}
\begin{itemize}
	\item HTTP traffic usually is very asymmetric
	\item a lot of the content is static
	\item network ACLs may restrict traffic flow
\end{itemize}
\vspace{.25in}
\begin{center}
	\includegraphics[scale=0.6]{pics/revproxy.eps}
\end{center}

\subsection{HTTP overload}
Ways to mitigate HTTP overload:

\begin{itemize}
	\item DNS round-robin to many web servers
	\item load balancing
	\item web cache / accelerators (reverse proxies)
	\item content delivery networks
\end{itemize}

These solutions depend on the location within the network and the scale of
the environment.

\subsection{Load Balancing}
\begin{center}
	\includegraphics[scale=0.55]{pics/Lb101.eps}
\end{center}

\subsection{Load Balancing: Inbound}
\begin{center}
	\includegraphics[scale=0.7]{pics/One-armed-inbound.eps}
\end{center}

\subsection{Load Balancing: Outbound}
\begin{center}
	\includegraphics[scale=0.7]{pics/One-armed-outbound.eps}
\end{center}

\subsection{Load Balancing: Direct Server Return}
\begin{center}
	\includegraphics[scale=0.6]{pics/DSR.eps}
\end{center}

\subsection{Content Delivery Networks}
\begin{center}
	\includegraphics[scale=0.9]{pics/cdn.eps}
\end{center}

\subsection{Content Delivery Networks}
\begin{itemize}
	\item cache content in strategic locations
	\item determine location to serve from via geomapping of IP
		addresses (beware IPv6 aggregation!)
	\item often uses a separate domain to distinguish small
		objects/large objects or dynamic content/static content
	\item either out-sourced or in-house (if your organization is a
		Tier-1 or Tier-2 peering partner)
	\item request routing happens via Global Server Load Balancing,
		DNS-based request routing, anycasting etc.
	\item provides vast amounts of interesting data about your clients
		(see \verb+https://www.akamai.com/stateoftheinternet/+)
\end{itemize}

\subsection{CDN Implications}
\begin{itemize}
	\item your CDN sees all your traffic
	\item your CDN controls your TLS certificate keys
	\item your CDN is a multi-tenant environment
	\item your CDN may impose restrictions on your clients
	\item separation of cache-able content may
		require multiple (second-level) domains
\end{itemize}

\subsection{HTTP and DNS}
\vspace*{\fill}
\Huge
Both HTTP and DNS are trivial to set up. \\

Both HTTP and DNS are not trivial to get right. \\
\Normalsize
\vspace*{\fill}

\subsection{Reading}
HTTP etc.:
\begin{itemize}
	\item RFC 2616, 2818, 3875
	\item \verb+https://httpd.apache.org/docs/+
	\item \verb+https://www.w3.org/Protocols/+
	\item REST: \verb+https://is.gd/leSvGa+
	\item CDNs: \verb+https://is.gd/R5DoxA+
		\begin{itemize}
			\item \verb+https://www.edgecast.com/+
			\item \verb+https://aws.amazon.com/cloudfront/+
			\item \verb+https://www.akamai.com/+
			\item \verb+https://www.limelight.com/+
			\item ...
		\end{itemize}
	\item \verb+https://developer.yahoo.com/performance/rules.html+
\end{itemize}


\end{document}
